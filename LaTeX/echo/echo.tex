% !TEX root = ../main.tex

\section{The Echo Effect}
We start out with the foundation for the project: The \emph{echo} effect. The echo effect is simply the delay of an audio signal, such that both the original and echo signal can be perceived in quick succession. We can experience this effect near large flat, hard surfaces. The large surface area results in a large scattering area, and the hard surface yields a low attenuation of the signal, such that we are able to perceive the echo above the background noise.

Echo is not limited to audio signals. In the field of radio signals, echoes can also be a observed. A GPS receiver can for instance experience echoes near tall buildings in cities, which decreases the accuracy. In this instance, it is desirable to limit the amplitude of the echoes, instead of creating them from a primary source, as in this project.

\subimport*{./}{theory.tex}
\subimport*{./}{implementation.tex}
\subimport*{./}{analysis.tex}
\subimport*{./}{results.tex}
