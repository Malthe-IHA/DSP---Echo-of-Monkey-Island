% !TEX root = ../main.tex

\subsection{Implementation}
We can generate the desired echo response in two filter ways: Either we can create a new track for the filtered output, and add it back to the master track, or we can do both steps in a single filter.

Since the second method is simpler, we have implemented this method. This results in a slight change of the coefficients from \cref{eqn:echocoeff}.
\begin{equation}
b_{0} = 1
\end{equation}
This creates a duplicate of the original track in our filtered output, so we do not have to do this manually. The resulting code for a simple script is the following:
\begin{listing}
\begin{minted}[linenos,breaklines,stepnumber=5,frame=single]{matlab}
% Import the audio.
[x, Fs] = audioread('hello.wav');  

% The number of unit step delay.
d = ceil(Fs / 1);

% The strenght of the echo.
alpha = 1;

% Generate FIR filter.
b = [1, zeros(1, d), alpha];

% Apply the FIR filter.
y = filter(b, 1, x);

% Write the output.
audiowrite('echo.wav', y, Fs);
\end{minted}
\caption{Simple MatLab script for experimenting with the audio delay. Intended for use with the sample from \cite{audiohello}.}
\label{lst:echosample}
\end{listing}

