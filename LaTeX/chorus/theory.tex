% !TEX root = ../main.tex

\subsection{Theory and design}
To generate a chorus effect we need to achieve 3 things with regards to the primary signal:
\begin{enumerate}
	\item Vary the delay of the signal, to simulate multiple sources. Must not be perceived as an echo.
	\item Alter the pitch of the signal, to emphasize the simulated multiple sources.
	\item Optionally alter the amplitude of the generated sources.
\end{enumerate}

The final point of amplitude modulation can be trivially done in digital signal processing, and we will not discuss this part further.

\subsubsection{Delay}
The first point of creating a signal which is offset from the primary signal by a slight delay is exactly what we have been doing in the first part of this report: The \emph{echo} effect. However, now the delay need to be so small that we do not perceive the tracks to be echoes of the primary signal, but rather additional sources of sound in tune with the primary signal.

Looking at the results from \cref{tbl:echo}, we aim for delays in the range of \SI{100}{\milli\second}. By applying a range of delays instead of just a fixed one, we simulate multiple sources.

\subsubsection{Pitch}
The second part of creating the chorus effect is changing the pitch of the duplicated tracks. There are several ways we can do this. One is the "correct" way, the other yields the same result due to the way filtering works and we perceive sound.
\begin{itemize}
	\item Change the frequency spectrum of the signal.
	\item Simulate changes by varying the delay.
\end{itemize}
We will start by examining the conventional way of changing the pitch: Altering the frequency components of the signal.

\paragraph{Frequency spectrum method}\mbox{}\\
Given a digital signal $x$ of $n$ samples, we can go from the time domain, where the signal was captured, to the frequency domain by applying the Fourier Transform (FT) to the signal. Since $x$ is not a mathematical function, but a series of quantized values, we are not able to use the continuous definition of the Fourier Transform, but have to apply the Discrete Fourier Transform (DFT) instead.

There are several ways to implement the DFT, however the naive way of summing the integrals in the FT results in a slow implementation that scales as $O(n^{2})$, where $n$ is the number of samples. The Fast Fourier Transform uses an alternative implementation, which let us do the calculations as $O(n \cdot \log(n))$ instead, which is significantly faster for large number of samples. \fxnote{Kunne skrive noget om $n$ her}

We now have a method to go from time to frequency domain, which yields the amplitude at each frequency component. This allows us to manipulate the signal contents in a different manner. For instance, we can now take the pitch of the signal, and move it up or down by shifting the entire frequency spectrum up or down. Should we wish for a less heavy handed approach, we can identify just the part of the signal we wish to shift, and do it selectively instead of altering the entire spectrum.

Once we are done with modifying the pitch, we can convert the frequency spectrum back into the time domain, using the Inverse Fourier Transform (IFT). As before, for quantized signals we need to use the discrete version (IDFT), and again optimized versions exist in the form of the FFT algorithm family (IFFT).

To summarize the procedure:
\begin{enumerate}
	\item Quantize signal.
	\item Transform from time domain to frequency domain.
	\item Alter frequency components to get desired pitch response.
	\item Transform back from frequency domain to time domain.
	\item "De-quantize"/playback the signal.
\end{enumerate}
The biggest issue with this method is the need to work on blocks of signal at a time, since we cannot stream the signal through the DFT.

\paragraph{Delay method}\mbox{}\\
  
%\fxnote{The following is WRONG}

An alternative to the frequency domain method mentioned above, is modulating the input signal as visualize in \cref{fig:chorus}. The figure shows two delay blocks with a length of $z^{-d}$ samples. At every sample time the newest input sample is fed to the left-hand side of the block, while the oldest sample is outputted at the right-hand side. This action defines the delay which is the same as described in the "echo" section above. The difference between the echo and chorus effect lies in the length of delay. Furthermore the chorus effect utilizes a varying delay.

To accomplish         
%, is to utilize %the human perception of hearing. Instead of doing the transformation to frequency space, we employ a number of delay lines in the form of digital filters, quite similar in design and effect as the echo filters we described earlier.

%This time, we utilize that the mixing effect of two sines. If we have two pure sine signals and play them on top of each other, we would get the following:
%\begin{equation}
%\sin(\omega_{A}) + \sin(\omega_{B}) = 2 \cdot \sin \left( \frac{\omega_{A} + \omega_{B}}{2} \right) \cdot \cos \left( \frac{\omega_{A} - \omega_{B}}{2} \right)
%\label{eqn:chorusbeat}
%\end{equation}
%Thus, we have a term that is proportional to the sum of the input frequencies, and one which is proportional to the difference. If the two frequencies are close, such that $\omega_{A} - \omega_{B}  = \omega_{\textrm{beat}} \ll 1$, we can rewrite it as
%\begin{equation}
%\sin(\omega_{A}) + \sin(\omega_{B}) = 2 \cdot \sin (\omega_{A}) \cdot \cos \left( \frac{\omega_{\textrm{beat}}}{2} \right)
%\label{eqn:chorusbeatsimple}
%\end{equation}
%We can simulate this effect using the delay lines available to us in a FIR filter. If we modulate the number of unit delays with the beat frequency $\omega_{\textrm{beat}}$
